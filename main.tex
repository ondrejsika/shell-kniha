\documentclass[12pt,a5paper]{article}
\usepackage[utf8]{inputenc}
\usepackage[margin=2cm]{geometry}
\usepackage[parfill]{parskip}
\usepackage{listings}
\lstset{
   breaklines=true,
   basicstyle=\ttfamily}

\title{Uvod do shellu}
\author{Ondrej Sika}
\date{3. 3. 2015}

\begin{document}
\maketitle

\newpage

$$$$

\vfill

{\LARGE Uvod do shellu}
\vspace{0.3cm}

{\large Ondrej Sika}\\
\texttt{ondrej@ondrejsika.com}\\
\texttt{http://ondrejsika.com}
\vspace{0.8cm}

Domovska stranka knihy je\\
\texttt{https://ondrejsika.com/books/uvod-do-shellu}

\newpage

$$$$

\tableofcontents



\section{Predmluva}

Pred tim nez jsem zacal vysvetlovat programovani musel jsem nejdrive vysvetlit jak se pracuje v terminalu, tedy v shellu. V teto knizce nechci popisovat jak funguji podminu a cykli v Bashi a jak v nem programovat, ale chci lidem ktery nemaji s shellem zadne zkusenosti ukazat jak shell funguje a jak v nem spoustet dalsi programy.

Tato knizka je vhodna pro uvod do Linuxu a je vhodna pred tim nez se pustime do programovani podle Python knihy a Flask knihy.

Pyhton kniha sice popisuje nejake zakladni funkce shellu, ale je to jen nutne minimum, pro spusteni Python shellu.

Preji prijemne cteni.


\section{Uvod}
\subsection{Co je to Shell?}

Shell je interpret prikazu ktery se stata o spoustneni programu, komunikaci s uzivartelem a jadrem operacniho systemu. \cite{wiki-shell}

\section{Druhy shellu}
\section{Bash}
\section{Souborovy system}
\section{Vim}
\section{Zaver}

\begin{thebibliography}{10}
%  \bibitem{python-kniha}Ondrej Sika:
%    {\em Python kniha}.\\
%    Sika Press, Praha, 2015.\\
%    ISBN 80-85867-35-4.\\
%    \lstinline|https://ondrejsika.com/books/python-kniha|
%  \bibitem{homesim}Ondrej Sika:
%    {\em Latex paper examples}. (brezen 2015).\\
%    \lstinline|https://ondrejsika.com/others/latex/|
  \bibitem{wiki-shell} Wikipedia:
    \emph{Shell (Programovani)}. (brezen 2015).\\
    \lstinline|https://cs.wikipedia.org/wiki/Shell_(programov%C3%A1n%C3%AD)|
\end{thebibliography}
\end{document}

